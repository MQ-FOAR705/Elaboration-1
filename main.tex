\documentclass[a4paper,12pt]{article}
\usepackage[english]{babel}
\usepackage[utf8]{inputenc}

%
% For alternative styles, see the biblatex manual:
% http://mirrors.ctan.org/macros/latex/contrib/biblatex/doc/biblatex.pdf
%
% The 'verbose' family of styles produces full citations in footnotes, 
% with and a variety of options for ibidem abbreviations.
%
\usepackage{csquotes}
\usepackage[style=verbose-ibid,backend=bibtex]{biblatex}
\bibliography{sample}

\usepackage{lipsum} % for dummy text

\title{Elaboration 1}

\author{John Hundley}

\date{\today}

\begin{document}
\maketitle

\section{Step-by-Step Breakdown of Revised Method:}

This document includes a step-by-step breakdown of the method as indicated in the computational analysis.

\begin{itemize} 

\item 1.	Identify thesis or argument. 
\item 2.	Identify relevant sources. 
\item 3.	Search necessary data bases. This includes Google Scholar, Macquarie University Library, NewCat (Newcastle University Database), JSTOR, and Trove. 
\item 4.	Access and download required sources. This would require access to aforementioned databases. Sometimes this requires a university enrolment ID or a subscription service. 
\item 5.	Use Zotero toolbar when accessing files to record and manage bibliography.
\item 6.	Store sources in files kept on Couldstor and OneDrive. 
\item 7.	Open Voyant and upload sources in search engine. Perform search. 
\item 8.	Identify key themes, terms, and connections as demonstrated in Voyant. This is indicated in the tables, graphs, and charts that Voyant provides. 
\item 9.	Collate this information and any metadata in a Word document or Cloudstor file. If necessary, record time stamps on toggl (time stamp tool). Save work on Cloudstor. 
\item 10.	Read documents on either a PDF view or printed.
\item 11.	Write response on Word or Cloudstor. Work to be saved to computer hard-drive and Cloudstor folder. 
\item 12.	Use Zotero to generate bibliography. 

\end{itemize}


\end{document}